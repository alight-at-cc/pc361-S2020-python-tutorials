\documentclass[12pt,letterpaper]{article}
\usepackage[margin=0.75in,]{geometry}
\usepackage{graphicx}
\usepackage{multirow}
\usepackage{hyperref}
%\usepackage[pdftex]{graphicx}
\pagestyle{plain}

\begin{document}

\pagenumbering{gobble}
\setlength\parindent{0mm}

\newcommand{\myparskip}{\vspace{2mm}}
\newcommand{\headingskip}{\vspace{1mm}}
%\begin{center}
%\LARGE  PHYS 017 Lab Syllabus\\
%\vspace{2mm}
%\normalsize Spring, 2018\\
%%\rule{0.9\textwidth}{0.1pt}
%\end{center}

\begin{tabular}{ l l }
  \multirow{4}{*}{\includegraphics[height=1in,]{swarthmore_logo}} 
  %\multirow{3}{*}{\includegraphics[height=1in,]{slothmore_logo.eps}} 
  &\\
  \headingskip
  & \large PHYS 017 Lab \\
  \headingskip
  & \large Practical Computing for Physicists \\
  \headingskip
  & \large M 1:15pm -- 4:15pm or 7:30pm -- 10:30pm\\
\end{tabular}

\begin{center}
  \rule{0.9\textwidth}{0.1pt}
\end{center}

% Professor information
\begin{center}
\setlength{\tabcolsep}{18pt}
\begin{tabular}{ l l l } 
  Instructors: & Paul Jacobs & Adam Light \\
& pjacobs3@swarthmore.edu & alight2@swarthmore.edu \\
& SC L67   & SC L40 \\
& (610) 328-8248   & (610) 328-6825 \\
  %\multicolumn{3}{p{4.5in}}{This syllabus will evolve as we together craft a course that provides opportunities for each of us to grow and to demonstrate growth.}\\\\
\end{tabular}

  \rule{0.9\textwidth}{0.1pt}
\end{center}

\textbf {\large Course Description:}  
The lab for PHYS 017 is a separate unit designed to give you some exposure to practical computation in physics.  
You will practice implementing and using typical data analysis and numerical solution methods in the Python scripting language.\\

\textbf {\large Course Structure and Workload:}
There will be one lab meeting each week, starting the second week of classes.  
During each session, you will work through a set of exercises that help you to practice certain skills.
You will use an electronic notebook for compiling Python scripts (Jupyter notebook), either on your own computer or on a lab computer.  
Although our expectation is that you will finish most of the work during the lab period, you will have up to 48 hours to complete any remaining exercises.
Grades will be based primarily on the code you submit in your Jupyter notebooks, which you will upload to Moodle (.ipynb file). \\

\textbf{\large Expected Background:} Although no physics is formally required, familiarity with basic mechanics and E\&M are helpful.  Some familiarity with the idea of writing a computer program is helpful but not absolutely necessary. 
\myparskip

If you need a refresher or would like to have some extra practice with Python, there are many online tutorials. 
\emph{A Student's Guide to Python for Physical Modeling} (on reserve at Cornell) might also be useful.\\

\textbf {\large Grading:}  
Your evaluation in this course will be based on your engagement in the lab community and on your Jupyter notebooks.  
Our hope is to impart some good tools and practices, most of which are captured by Greg Wilson et al. in \href{https://journals.plos.org/plosbiology/article?id=10.1371/journal.pbio.1001745#s1a}{\textit{Best Practices for Scientific Computing}}.
Each week, your work will be assessed using the following rubric:

\begin{itemize}
	\item (3 points) Functionality: Code meets assignment specifications, runs without error, and includes example function calls demonstrating its use.

	\item (3 points) Documentation: Docstrings are included for all functions, and inline comments are included where necessary for clarity. Documentation should explain purpose and design choices rather than operation of code. The documentation should explain the inputs that functions expect and the meaning of outputs rather than internal mechanics.  

	\item (3 points) Clarity: Code demonstrates that thought has been given to modularity, efficiency, and transparency. Code is readable, without overly concise or convoluted constructions.  Code is not repeated unnecessarily.  Function and variable names are consistent, distinctive, and meaningful. 

	\item (1 points) Engagement: Engagement is assessed as “full” or “none” for each lab session based your conscientious participation in lab activities. Activities include individual work, group discussion, and one-on-one conversations with other students. If you are absent, asleep, engaged in activities unrelated to class, or destructive to the classroom community, you may earn a zero for that day.
	
\end{itemize}

\textbf {\large Late Work:}
Notebooks turned in after the 48-hour deadline will receive a maximum score of 5/10.
No notebooks will be accepted beyond the start of the next week's lab.\\

\textbf {\large Lab Topic Schedule}:

\begin{table}[h!]
\fontsize{11}{13} % The size of the table text can be changed depending on content. Remove if desired.
\selectfont
\begin{tabular}{ | c | c |}
\hline
\textbf{Week} & \textbf{Content} \\
\hline
\begin{minipage}{0.15\textwidth}Lab 0 \\ 1/21 \end{minipage} & 
\begin{minipage}{.8\textwidth}
\begin{itemize} \itemsep-0.4em
	\vspace{1mm}
	\item Tutorial/Introduction: functions, operations, and plotting (not graded)\\
	\vspace{1mm}
\end{itemize}
\end{minipage} \\
\hline
\begin{minipage}{0.15\textwidth}Lab 1 \\ 1/28 \end{minipage} & 
\begin{minipage}{.8\textwidth}
\begin{itemize} \itemsep-0.4em
	\vspace{1mm}
	\item Numerical integration\\
	\vspace{1mm}
\end{itemize}
\end{minipage} \\
\hline\begin{minipage}{0.15\textwidth}Lab 2 \\ 2/4 \end{minipage} & 
\begin{minipage}{.8\textwidth}
\begin{itemize} \itemsep-0.4em
	\vspace{1mm}
	\item Iterative methods for solving ODEs\\
	\vspace{1mm}
\end{itemize}
\end{minipage} \\
\hline
\begin{minipage}{0.15\textwidth}Lab 3 \\ 2/11 \end{minipage} & 
\begin{minipage}{.8\textwidth}
\begin{itemize} \itemsep-0.4em
	\vspace{1mm}
	\item Monte Carlo simulation\\
	\vspace{1mm}
\end{itemize}
\end{minipage} \\
\hline
\begin{minipage}{0.15\textwidth}Lab 4 \\ 2/18 \end{minipage} & 
\begin{minipage}{.8\textwidth}
\begin{itemize} \itemsep-0.4em
	\vspace{1mm}
	\item FFTs and filtering\\
	\vspace{1mm}
\end{itemize}
\end{minipage} \\
\hline
\begin{minipage}{0.15\textwidth}Lab 5 \\ 2/25 \end{minipage} & 
\begin{minipage}{.8\textwidth}
\begin{itemize} \itemsep-0.4em
	\vspace{1mm}
	\item Solving linear systems, curve-fitting\\	
	\vspace{2mm}
\end{itemize}
\end{minipage} \\
\hline
\begin{minipage}{0.15\textwidth}
	\vspace{1mm}Lab 6 \\ 3/4 \end{minipage} & \
\begin{minipage}{.8\textwidth}
\begin{itemize} \itemsep-0.4em
	\vspace{1mm}
	\item Synthesis: gravitational wave detection
	\vspace{1mm}
\end{itemize}
\end{minipage} \\
\hline

\end{tabular} 
\end{table}

\end{document}

